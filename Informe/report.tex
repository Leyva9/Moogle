\documentclass{article}
\usepackage[spanish,activeacute]{babel}
\usepackage[utf8]{inputenc}
\usepackage{graphicx}
\usepackage{float}
\usepackage{soul}

\title{Informe del Proyecto Moogle en \LaTeX}
\author{Luis Manuel Leyva-Hernández}

\begin{document}
\begin{figure}
    \centering
    \includegraphics[width=\textwidth]{87f4e980-62a6-11eb-846f-4c58547cffc0.png}
    \label{main-img}
\end{figure}
\maketitle
\sethlcolor{gray}
\section*{Introducción}

Moogle! es una aplicación totalmente original cuyo propósito es buscar inteligentemente un texto en un conjunto de documentos. Es una aplicación web, desarrollada con tecnología $.NET$ Core 6.0, específicamente usando Blazor como framework web para la interfaz gráfica, y en el lenguaje C\#. La aplicación está dividida en dos componentes fundamentales:\\

\textbf{MoogleServer} es un servidor web que renderiza la interfaz gráfica y sirve los resultados.\\

\textbf{MoogleEngine} es una biblioteca de clases donde está implementada la lógica del algoritmo de búsqueda.\\

\section*{Desarrollo}

\title{\textbf{Intrucciones:}}\\


    \scriptsize \textbf{Nota:} La raíz donde se pondrán todos los documentos .txt es la siguiente \verb*|"moogle\MoogleServer\Content"| verifique que este copiando los documentos en esta ruta\\

    \normalsize
\title{\textbf{Ejecución}}
    \begin{itemize}
        \item Si vas a ejecutar el proyecto en \textbf{VS code} sigue estos pasos:
        \begin{enumerate}
            \item Abre Visual Studio Code y selecciona "Archivo" > "Abrir carpeta" para abrir la carpeta del
            proyecto.
            \item \textbf{Instala la extensión de C\#:} Si aún no lo has hecho, debes instalar la extensión de C\# para Visual
            Studio Code. Puedes hacerlo seleccionando "Extensiones" en el menú lateral, buscando "C\#" y
            seleccionando "Instalar".
            \item \textbf{Configura el entorno de ejecución:} Si el proyecto requiere una configuración específica, como un
            archivo de configuración o una variable de entorno, asegúrate de configurar el entorno
            adecuadamente antes de ejecutar el proyecto. 
            \item \textbf{Ejecuta el proyecto:}  Una vez que hayas instalado la extensión de C\# y configurado el entorno
            adecuadamente, puedes ejecutar el proyecto utilizando la terminal integrada de Visual Studio Code.
            Abre la terminal en Visual Studio Code seleccionando "Terminal" > "Nueva terminal" en el menú.
            Luego, ejecuta el siguiente código desde la terminal: dotnet watch run --project MoogleServer
        \end{enumerate}
        \item Si vas a ejecutar el proyecto en \textbf{Visual Studio IDE} sigue estos pasos:
        \begin{enumerate}
            \item Abre Visual Studio y selecciona "Archivo" > "Abrir" > "Proyecto/Solución" para abrir el
            archivo de proyecto de C\#.
            \item Antes de ejecutar el proyecto, debes compilarlo para asegurarte de que no hay errores de
            compilación. Puedes compilar el proyecto seleccionando "Compilar" > "Compilar solución"
            desde el menú.
            \item Una vez que hayas compilado el proyecto y configurado el proyecto de inicio adecuado, puedes
            ejecutar el proyecto seleccionando "Depurar" > "Iniciar depuración" desde el menú. Esto
            compilará el proyecto si es necesario y lo ejecutará en el depurador de Visual Studio.
        \end{enumerate}
    \end{itemize}
\title{\textbf{Fase Prueba}}
    \begin{itemize}
        \item \textbf{Introducción de un término:}
            \begin{enumerate}
                \item Para buscar introduzca algún o algunos términos en la caja de texto que tiene. Se asegura de
                que si no se inserta nada no se realiza ninguna búsqueda y se retorna una alerta.
                \item Cualquier término que inserte puede ser escrito con tildes o sin ellas, con mayúsculas o
                minúsculas, esto no afectará la búsqueda de esa palabra.
            \end{enumerate}
        \item \textbf{Retornando documentos:}
            \begin{enumerate}
                \item Una vez introducido el término o los términos haga click en el botón buscar y se mostrará
                por orden de relevancia los documentos de mayor puntuación a las menores puntuaciones.
                \item Si ninguno de los términos introducidos no están contenidos en ningún documento de texto
                de nuestra base de datos, entonces el retorno será nulo.
                \item Los resultados serán devueltos con el siguiente modelo, Título (...Snippet...)
                
                \includegraphics[width=0.85\textwidth]{87f4e980-62a6-11eb-846f-4c58547cffc0.png}

                \item Se garantiza para futura implementación que cada título de cada documento mostrado tenga
                un hipervínculo para acceder en caso de tener un servidor local montado.

                \item Para garantizar una imagen limpia en el programa se decide mostrar los mejores cinco
                documentos, en caso de no haber cinco se garantiza que se muestren solo los que tuvieron
                relevancia.

                \item Dese cuenta que el Snippet mostrado de cada documento es aleatorio, para probarlo solo
                basta con presionar de nuevo el botón buscar con el mismo término y notará que el texto
                contenido por cada documento mostrado cambiará.

                \item Este procedimiento de búsqueda lo puede repetir indefinidamente y el tiempo de espera
                entre búsquedas es lo suficientemente rápido para garantizar una buena experiencia.
                
            \end{enumerate}
    \end{itemize}
\title{\textbf{Implementación:}}
    \begin{itemize}
        \item \textbf{Clases utilizadas:}
        \begin{enumerate}
            \item class \textbf{Indexer:} Esta clase contiene el cuerpo principal de la base de datos dada, algunos de
            sus principales métodos son:
                \begin{itemize}
                    \item El método \hl{private void ReturnWords(string FilesPath)} es el encargado de hacer una lista de
                    diccionarios donde se almacenan todas las palabras sin repetir por documento de todos
                    los documentos.\\

                    \scriptsize \textbf{Nota:} Este método aprovecha la iteración principal para calcular el TF de cada palabra de un documento
                    en cada ciclo de la iteración principal, y dentro de la iteración secundaria se aprovecha para ir haciendo
                    DF de cada palabra de la siguiente manera:

                    
                        \includegraphics[width=0.85\textwidth]{87f4e980-62a6-11eb-846f-4c58547cffc0.png}
                    
                    \normalsize
                    \item El método \hl{public static string NormalizeTheText(string fileText)} normaliza
                    el texto que recibe quitando los signos de puntuación y se declara estático para que este
                    pueda ser usado fuera de la clase, por ejemplo en el query del usuario.

                    \item Este método \hl{public static string GetDocumentSnippet(string file, int
                    snippetLength = cant-lineas-fragmento)} toma un fragmento aleatorio del
                    documento y lo devuelve para que luego pueda ser mostrado como Snippet.
                    
                \end{itemize}

            \item class \textbf{Vector}: Esta clase obtiene el score final de los vectores de los Documentos por el
            vector Query y algunos de sus principales métodos son:
                \begin{itemize}
                    \item El método \hl{private double VectorDotProduct(double[] vectorA, double[]
                    vectorB)} realiza el producto punto entre dos vectores y devuelve el valor expresado
                    en double. Se asegura que los vectores tengan el mismo tamaño y no sean nulos.

                    \item El método \hl{private double VectorMagnitudeProduct(double[] vectorA,
                    double[] vectorB)} calcula el producto de la magnitud entre dos vectores y
                    devuelve el valor expresado en double. Se asegura que los vectores tengan el mismo
                    tamaño y no sean nulos.

                    \item El método \hl{public static int[] GetTopFiveDocumentScore(double[]
                    documentScore)} Este método obtiene los cinco documentos con mayor relevancia
                    en cuanto al score, se asegura que en caso de que la cantidad de documentos en la
                    base de datos es menor que cinco se cree un array de tamaño del mínimo entre
                    documentScore.Lenght o 5, asegurando que no se salga de los límites.
                \end{itemize}
        \end{enumerate}
    \end{itemize}
    \section*{Conslusiones}
    \begin{itemize}
        \item En conclusión, el software de búsqueda de palabras clave ha demostrado ser una herramienta altamente beneficiosa para la extracción de información relevante en archivos de texto, y su implementación promete mejorar la eficiencia y productividad en el manejo de grandes volúmenes de datos de texto.
        \item Tras realizar un exhaustivo análisis y evaluación del software de búsqueda de palabras clave en archivos de texto, se han obtenido varios hallazgos significativos. En general, el software demostró ser una herramienta efectiva para identificar palabras clave y términos relevantes en los documentos analizados. Sus algoritmos de búsqueda y filtrado mostraron resultados precisos y presentaron las ocurrencias de palabras clave de manera clara y concisa.
    \end{itemize}
    
\end{document}